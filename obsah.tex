%=========================================================================
% (c) Michal Bidlo, Bohuslav Křena, 2008

\chapter{Úvod}
V minulosti v oblasti herného priemyslu existovalo niekoľko poskytovateľov, ktorý sprostredkovali rôzne služby pre vývojárov hier, ako napríklad tabuľky online hráčov, správa odmien, pokročilé štatistiky prístupov, alebo vlastné štatistiky. Tieto služby boli orientované iba na webové flash hry a aj väčšina z tých pomaly zaniká, alebo už zanikla pomaly upadajúcou technológiu flash. Za zmienku stojí napríklad leader sveta flashových hier menom Mochimedia. S rozmachom moderných webových technológií a príchodom HTML5/Javascript hier dlho neexistovala žiadna webová stránka ponúkajúca podobné vyššie zmienené služby. Príchodom webového portálu clay.io v roku 2012 sa vývojárov javascriptových hier usmialo štastie. Clay.io ponúkalo rôzne služby pre herných vývojárov, bez nutnosti mať státisíce prístupov do hry, alebo mať hru špičkovej úrovni. Služby boli dostupné pre všetkých bez rozdielu. Avšak začiatkom roka 2015 sa clay.io začalo orientovať iným smerom a z tejto webovej služby sa stal viac portál pre hráčov ako pre vývojárov. Už viac nebolo možné integrovať API, do hociktorej hry a všetky hry, ktoré chceli byť umiestnené na portáli clay.io a využívať jeho služby boli vyberané podľa kvality. Taktiež hry využívajúce API bolo možné hrať iba na webe clay.io a developer nesmel distribuovať hru na iných webových portáloch. Týmto sa clay.io uzavrelo a zaradilo sa vedľa iných svetových distribútorov HTML5 hier, ako napríklad Boostermedia, Softgames, alebo Spilgames. 
Táto bakalárska práca vznikla z dôvodu neexistenice otvoreného modernej webovej platformy na tvorbu hier pre všetkých bez rozdielu a služby ponúkajúc zadarmo. Táto webová platforma by mala byť orientovaná nielen na vývojárov javascriptových hier, ktorý prostredníctvom API by mali možnosť využívať služby platformy, ale aj na hráčov týchto vytvorených hier. Hráči by mali byť schopný hrať hry na akejkoľvek platforme, či už desktopových osobných počítačoch, alebo moderných chytrých telefónoch. Z toho dôvody by mala minimálne časť webového portálu určená hráčom byť do určitej miery responzivna.
Nasledujúca kapitola sa venuje predstaveniu technológiam, ku ktorým sa dostalo a následne boli využité v rámci vypracovania tejto práce. Ďalšia kapitola sa venuje návrhu celého webového portálu a API. Po návrhu prišla na rad implementácia, o ktorej je písané v rovnako pomenovanej kapitole. A nakoniec celá webová platforma bola zverejnená a testovaná v reálnych podmienkach. O tejto fázi hovorí kapitola testovanie.

\chapter{Použité technológie}
Pri implementácií platformy bolo nutné si dobre premyslieť čo bude potreba a špecifikovať požiadavky na technológie, ktoré sa budú využívať. Keď sa zhrnuli všetky požiadavky, tak z toho vyšlo, že je potreba nejaký programovací jazyk pre serverovú časť, klientskú časť, jazyk pre API pre vývojárov, databáza a databázový jazyk a jazyk, v ktorej bude napísaná hra demonštrujúca funkcie webovej platformy.

\section{Serverová časť}
Pri výbere jazyka pre serverovú časť sa ponúkalo hneď niekoľko možností. Bolo na výber medzi PHP, Python, Java, Ruby, alebo ASP . NET. Zo všetkých možostí bol nakoniec vybraný skiptovací jazyk PHP, z toho dôvodu, že je to je najviac rozšírený serverový jazyk a väčšina poskitovateľov ponúka hosting práve s nainštalovaným PHP. 

\subsection{PHP}
Hypertext Preprocesor je open source skriptovací jazyk, používaný hlavne pre programovanie klient-server aplikácií a pre vývoj dynamických webových stránok.

\subsection{Nette}
dsfdsfdssdfds

\section{Databázová časť}

\subsection{MySQL}

\section{Klientská časť}

\subsection{HTML a CSS}

\subsection{Javascript}

\subsection{Bootstrap}

\section{API}

\subsection{AJAX}

\subsection{CORS}

\section{Hra}

\subsection{Phaser}

\chapter{Návrh riešenia}

\chapter{Implementácia}

\chapter{Testovanie}

\chapter{Záver}












%=========================================================================
